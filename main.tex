\documentclass{beamer}
\usepackage[utf8]{inputenc}
\usepackage{setspace}
\usepackage{times}
\usepackage{url}
\usepackage{longtable}
\urlstyle{same}
\usepackage{multirow}
% Useful for checking layout
% \usepackage{showframe}
% Set language
\usepackage[german]{babel}
\usepackage{booktabs}
\usepackage[scale=2]{ccicons}
% Abbreviations
\usepackage{glossaries}
% For big tables which span multiple pages
\usepackage{longtable}
% Images
\usepackage{graphicx}
\graphicspath{ {./images/} }
% APA Citation Style
\usepackage[natbibapa]{apacite}

\makeglossaries

\newglossaryentry{covid19}
{
    name=Covid-19,
    description={Coronavirus-Krankheit-2019 ~\citep{covid19}}
}
\newglossaryentry{foph}
{
    name=FOPH,
    description={Federal Office of Public Health}
}
\newglossaryentry{bfs}
{
    name=BFS,
    description={Bundesamt für Statistik}
}
\newglossaryentry{who}
{
    name=WHO,
    description={World Health Organization}
}
\newglossaryentry{crisis_visualization}
{
    name=Crisis Visualization,
    description={Visuelle Repräsentation von Informationen über unerwünschte Situationen, welche eine Bedrohung für die Menschheit darstellen. \citep[S. 2]{mapping_landscape_of_covid19_crisis_visualizations}}
}



\usetheme[progressbar=foot, numbering=counter, background=dark]{metropolis}
\title{Analyse und Evaluation von Corona-Datenvisualisierungen}
\date{\today}
\author{Yannick Hutter}
\institute{Fachhochschule Graubünden - Digital Business Management}
\begin{document}

\maketitle


\section{Begründung der Relevanz}
\begin{frame}
    \begin{center}
        {\huge \textbf{3'000'000 erkrankte Personen}}\\
        Im Zeitraum vom 24.02.2020 bis zum 10.03.2022 (Schweiz + Liechtenstein) ~\citep{covid19_laboratory_confirmed_cases}
    \end{center}
\end{frame}

\begin{frame}
    \begin{center}
        {\huge \textbf{Datenvisualisierungen für}}
        \begin{itemize}
            \item Sensibilisierung
            \item Überwachung / Gefahrenerkennung
            \item Förderung vom gemeinsamen Verständnis
        \end{itemize}
    \end{center}
\end{frame}

\section{Stand der Forschung}
\begin{frame}{Definition von Datenvisualisierungen}
    \begin{figure}[ht]
        \includegraphics[width=10cm]{definition_data_visualization.png}
        \centering
        \caption{Definition Begriff Datenvisualisierung ~\citep[S. 12]{datenvisualisierungen_modern_web}}
    \end{figure}
\end{frame}

\begin{frame}{Mapping the Landscape of Covid19 Crisis Visualizations}
    \begin{figure}[ht]
        \includegraphics[width=8cm]{mapping_the_landscape_of_covid19_crisis_visualizations.png}
        \centering
        \caption{Mapping the Landscape of Covid19 Crisis Visualizations  ~\citep[S. 1]{mapping_landscape_of_covid19_crisis_visualizations}}
    \end{figure}
\end{frame}

\begin{frame}{Mapping the Landscape of Covid19 Crisis Visualizations}
    \begin{figure}[ht]
        \includegraphics[width=10cm]{codebook_sample.png}
        \centering
        \caption{Ausschnitt Codebook ~\citep{mapping_landscape_of_covid19_codebook}}
    \end{figure}

    \begin{itemize}
        \item 668 Visualisierungen untersucht, kategorisiert und codiert
        \item Codebook  erstellt (siehe Abbildung 3)
        \item Erstellung eines Modells zum Verständnis von Crisis Visualizations
    \end{itemize}
\end{frame}

\begin{frame}{Mapping the Landscape of Covid19 Crisis Visualizations}
    \begin{figure}[ht]
        \includegraphics[width=10cm]{conceptual_framework_mapping_landscape_of_covid19.png}
        \centering
        \caption{Modell zum Verständnis von Crisis Visualizations  ~\citep[S. 2]{mapping_landscape_of_covid19_crisis_visualizations}}
    \end{figure}

    \begin{itemize}
        \item Basiert auf dem Kommunikationsmodell von Lasswell
        \item Paper behandelte die ersten vier Schritte
    \end{itemize}
\end{frame}

\section{Forschungsfrage}
\begin{frame}

    \begin{center}
        \textbf{Wie nimmt die allgemeine Bevölkerung Corona Datenvisualisierungen wahr?}
    \end{center}
\end{frame}

\section{Methodische Vorgehensweise}
\begin{frame}{Methodik}
    \textbf{Wie sind bestehende Studien vorgegangen?}
    \begin{itemize}
        \item Mixed Method Ansatz (Heuristische Evaluation + Fragenbasierender Ansatz) ~\citep{evaluating_information_visualization}
        \item Quantitativ + Qualitativ (CUE / meCUE + Fragebasierender Ansatz) ~\citep{measuring_ux_of_data_visualization}
    \end{itemize}

    \begin{figure}[ht]
        \includegraphics[width=7cm]{meCUE_modell.png}
        \centering
        \caption{meCUE Modell ~\citep{meCUE_model}}
    \end{figure}
\end{frame}

\begin{frame}{Methodik}
    \textbf{Wie möchte ich bei meiner Arbeit vorgehen?}

    \begin{itemize}
        \item Qualitative Erhebungsmethode
        \item Auswahl von Repräsentanten gemäss der Studie von Zhang
        \item Nachbau der Grafik mit bestehendem Datensatz
        \item Erstellung von verschiedenen Darstellungsformen (Original, Verletzung der Design Prinzipien, Verbesserung durch Befolgung der Design Prinzipien)
        \item Erstellung einer Website welche die Darstellungen randomisiert anzeigt
        \item Durchführung einer Heuristischen Evaluation / Fragebasierender Ansatz
    \end{itemize}
\end{frame}



\begin{frame}{Aufbau}
    \begin{figure}[ht]
        \includegraphics[width=4cm]{quantitative_qualitative_methods.png}
        \centering
        \caption{Ablauf quantitatives vs. qualitatives Vorgehen}
    \end{figure}
\end{frame}
\section{Zeitplan}
\begin{frame}
    \begin{table}
        \begin{tabular}{@{} lr @{}}
            \toprule
            Aufgabe                                    & Deadline   \\
            \midrule
            Exposé überarbeiten                        & 07.04.2022 \\
            Fertigstellung theoretische Fundierung     & 14.04.2022 \\
            Auswahl von Corona Datenvisualisierungen   & 14.04.2022 \\
            Fertigstellung Definition von Begriffen    & 21.04.2022 \\
            Erstellung von Visualisierungen (Website)  & 28.04.2022 \\
            Rekrutierung von Probanden (5-8 Personen)  & 28.04.2022 \\
            Fertigstellung Konzeption der Untersuchung & 07.05.2022 \\
            \bottomrule
        \end{tabular}
        \caption{Zeitplan (Eigene Darstellung)}
    \end{table}
\end{frame}

\begin{frame}
    \begin{table}
        \begin{tabular}{@{} lr @{}}
            \toprule
            Aufgabe                              & Deadline   \\
            \midrule
            Erstellung des Erhebungsinstruments  & 07.05.2022 \\
            Durchführung der Studie              & 28.05.2022 \\
            Abgabe Exposé                        & 17.06.2022 \\
            Auswertung der Studie                & 18.06.2022 \\
            Fertigstellung Ergebnisdarstellung   & 25.06.2022 \\
            Fertigstellung Ausblick / Diskussion & 25.06.2022 \\
            Korrektur Arbeit                     & 25.06.2022 \\
            Abgabe Arbeit                        & 25.07.2022 \\
            \bottomrule
        \end{tabular}
        \caption{Zeitplan - Fortsetzung (Eigene Darstellung)}
    \end{table}
\end{frame}

\section{Fragediskussion}
\begin{frame}
    \textbf{Eigene Fragen}
    \begin{itemize}
        \item Soll die Studie nur auf einen Teilbereich der Nachrichtentypen von Zhang reduziert werden (Top 3)?
        \item Soll evtl. ein Remote User Testing durchgeführt werden (Maze, Hotjar)?
    \end{itemize}
\end{frame}

\appendix
\begin{frame}[allowframebreaks]{Anhang}
    \bibliographystyle{apacite}
    \urlstyle{rm}
    \bibliography{kolloqium_presentation.bib}
    \printglossary
\end{frame}
\end{document}